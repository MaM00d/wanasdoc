 ------------------------------------------------------------------------------
% Chapter 2

\chapter{Related Work}
\section{Introduction}
In this chapter, we will briefly explore some prior works related to our project, focusing on mental health assistance applications and AI-driven chatbots designed for user interaction.

\section{Similar Apps}
Several applications bear similarities to ours. By examining these, we identified the necessary requirements and designed our application accordingly.

\subsection{Talkspace}
Talkspace provides therapy through text, audio, and video sessions with licensed therapists. It has made therapy more accessible, though it still faces challenges related to cost and availability. Talkspace offers:
\begin{itemize}
    \item **Online Therapy**: Users can communicate with therapists via text, audio, and video messages.
    \item **Live Sessions**: Real-time therapy sessions through video calls.
    \item **Specialized Therapists**: Therapists who specialize in various mental health issues such as anxiety, depression, and PTSD.
    \item **Secure Messaging**: Ensures all communications are secure and confidential.
    \item **Comprehensive Resources**: Articles, worksheets, and self-help tools to support users between therapy sessions.
\end{itemize}

\subsection{Woebot}
Woebot is an AI chatbot offering cognitive-behavioral support by interacting with users, tracking their moods, and providing daily support. Key features include:
\begin{itemize}
    \item **AI Chatbot**: Engages users in conversations using CBT techniques to provide coping strategies.
    \item **Mood Monitoring**: Tracks users' moods and provides feedback based on mood data.
    \item **Psychoeducation**: Educates users about mental health and CBT techniques.
    \item **Skill-Building**: Guides users through exercises to develop coping skills.
    \item **Personalization**: Tailors recommendations and resources to users' needs.
\end{itemize}

\subsection{Wysa}
Wysa is an AI chatbot that provides mental health support using techniques from cognitive-behavioral therapy (CBT), dialectical behavior therapy (DBT), and mindfulness. Its functionalities include:
\begin{itemize}
    \item **AI Chatbot**: Empathetic conversations using natural language processing.
    \item **Self-Help Tools**: Tools and techniques from CBT, DBT, and mindfulness.
    \item **Mood Tracking**: Insights and patterns from tracking moods and emotions.
    \item **Personalized Support**: Tailored mental health resources and exercises.
    \item **Human Support**: Access to professional therapists if needed.
\end{itemize}

\section{Functionality}

\subsection{Wysa}
Wysa is an AI-driven chatbot designed to support mental health and well-being. It offers various functionalities to help users manage their mental health:

\begin{enumerate}
    \item \textbf{AI Chatbot:}
    \begin{itemize}
        \item Provides empathetic conversations using AI to help users talk through their feelings.
        \item Uses natural language processing to understand and respond to users' concerns.
    \end{itemize}
    \item \textbf{Self-Help Tools:}
    \begin{itemize}
        \item Offers tools and techniques based on Cognitive Behavioral Therapy (CBT), Dialectical Behavior Therapy (DBT), and mindfulness.
        \item Provides exercises and activities to help manage stress, anxiety, and depression.
    \end{itemize}
    \item \textbf{Mood Tracking:}
    \begin{itemize}
        \item Allows users to track their moods and emotions over time.
        \item Provides insights and patterns based on mood tracking data.
    \end{itemize}
    \item \textbf{Personalized Support:}
    \begin{itemize}
        \item Delivers personalized mental health resources and exercises tailored to users' needs.
        \item Includes goal-setting and progress tracking features.
    \end{itemize}
    \item \textbf{Human Support:}
    \begin{itemize}
        \item Offers access to professional therapists if users need human intervention.
        \item Provides a blend of AI-driven support and human counseling.
    \end{itemize}
\end{enumerate}

\subsection{Woebot}
Woebot is an AI-powered mental health chatbot that uses principles of CBT to help users manage their mental health. Its functionalities include:

\begin{enumerate}
    \item \textbf{AI Chatbot:}
    \begin{itemize}
        \item Engages users in conversation to help them process their emotions and thoughts.
        \item Uses CBT techniques to provide helpful responses and coping strategies.
    \end{itemize}
    \item \textbf{Mood Monitoring:}
    \begin{itemize}
        \item Tracks users' moods and mental health status over time.
        \item Provides feedback and suggestions based on mood data.
    \end{itemize}
    \item \textbf{Psychoeducation:}
    \begin{itemize}
        \item Educates users about mental health and CBT techniques.
        \item Offers informative content to help users understand their mental health better.
    \end{itemize}
    \item \textbf{Skill-Building:}
    \begin{itemize}
        \item Guides users through exercises and activities to develop coping skills.
        \item Uses evidence-based techniques to help users manage stress and anxiety.
    \end{itemize}
    \item \textbf{Personalization:}
    \begin{itemize}
        \item Adapts to users' needs and preferences over time.
        \item Provides tailored recommendations and resources.
    \end{itemize}
\end{enumerate}

\subsection{Talkspace}
Talkspace is an online therapy platform that connects users with licensed therapists. It offers several key functionalities:

\begin{enumerate}
    \item \textbf{Online Therapy:}
    \begin{itemize}
        \item Provides access to licensed therapists through text, video, and audio messages.
        \item Allows users to communicate with their therapists asynchronously.
    \end{itemize}
    \item \textbf{Live Sessions:}
    \begin{itemize}
        \item Offers live video therapy sessions with licensed therapists.
        \item Enables real-time interaction and counseling.
    \end{itemize}
    \item \textbf{Specialized Therapists:}
    \begin{itemize}
        \item Matches users with therapists who specialize in various mental health issues, such as anxiety, depression, and PTSD.
        \item Allows users to select therapists based on their specific needs.
    \end{itemize}
    \item \textbf{Secure Messaging:}
    \begin{itemize}
        \item Ensures all communications between users and therapists are secure and confidential.
        \item Uses encryption to protect users' privacy.
    \end{itemize}
    \item \textbf{Comprehensive Resources:}
    \begin{itemize}
        \item Provides a range of mental health resources, including articles, worksheets, and self-help tools.
        \item Offers continuous support and guidance between therapy sessions.
    \end{itemize}
\end{enumerate}

\section{Idea Inspiration}
as going through the previous apps that were mentioned and many other apps we thought to give out something to the community that can contribute to benefit the mental health of individuals, as such we took a brief look on the Talkspace app and how it gives the user the benefit of contacting actual therapists and the enviroment of the app is healing then we decided to look further and then we found Woebot and Wysa that implement AI as a chat bot to chat with the user and make it more friendly and more comfy to not have embarrassment when talking to another human being.

\section{Comparison}
Other apps compared to ours we decided to add more features than just a chatbot to chat with:
\begin{itemize}
    \item New personas every time the user wants to chat, enhancing comfort and engagement.
    \item A mood tracker that tracks daily mood and provides statistical insights over weeks, months, and years.
    \item Self-care features and daily challenges.
    \item Arabic language support for the chatbot, catering to Arabic-speaking users.
\end{itemize}

\section{Dependencies}
Our AI leverages existing models like BERT and AraBERT, fine-tuned to our requirements. 

\textbf{BERT (Bidirectional Encoder Representations from Transformers)} is a state-of-the-art natural language processing model developed by Google that understands context in both directions in a sentence, significantly improving performance on a wide range of language tasks.

\begin{figure}
    \centering
    \includegraphics[width=0.25\linewidth]{mo}
    \caption{AraBERT}
\end{figure}

\textbf{AraBERT} is a pre-trained language model specifically designed for Arabic text, based on BERT's architecture. It addresses the unique challenges of Arabic text, including complex morphology and diverse dialects. It is trained on large-scale Arabic corpora and fine-tuned for various NLP tasks, significantly improving the performance of Arabic NLP applications.

We also utilized datasets generated with ChatGPT for fine-tuning.
now are currently using:
\textbf{ACE GPT (Advanced Cognitive Engine for Generative Pre-trained Transformer)}

ACE GPT is an advanced language model specifically designed to enhance natural language understanding and generation tasks. It builds upon the architecture of OpenAI's GPT models, integrating additional features and optimizations to improve performance in various applications, including conversational AI, text completion, and content generation. ACE GPT leverages large-scale datasets and fine-tuning techniques to provide contextually relevant and coherent responses, making it a powerful tool for developing intelligent chatbots and other AI-driven applications.

\textbf{ACE GPT Chat}

ACE GPT Chat is a specialized extension of ACE GPT focused on providing interactive conversational capabilities. It is tailored for real-time dialogue, offering users engaging and contextually appropriate responses. ACE GPT Chat incorporates advanced natural language processing techniques to maintain the flow of conversation, understand user intent, and deliver personalized support. This makes it an ideal solution for applications in customer service, virtual assistants, and mental health support chatbots.

\section{Technical Related Work}
\textbf{Natural Language Processing (NLP) in Mental Health}

NLP is crucial for creating AI-driven mental health applications, enabling AI to understand and generate human-like responses, analyze emotions, and offer empathetic support. Key NLP techniques include:

\begin{itemize}
    \item \textbf{Sentiment Analysis}: This technique evaluates the emotional tone of user inputs, enabling AI to respond appropriately to various sentiments.
    \item \textbf{Emotion Detection}: It identifies specific emotions such as happiness, sadness, anger, and fear in user conversations, allowing AI to tailor its responses.
    \item \textbf{Conversational Agents}: These chatbots simulate human conversation, employing NLP techniques to understand queries, provide relevant responses, and engage in meaningful dialogues.
    \item \textbf{Text Classification}: This involves categorizing user inputs into predefined classes or topics, helping the AI to understand the context and intent behind the user's message.
    \item \textbf{Named Entity Recognition (NER)}: Identifying and classifying key information (names, dates, places, etc.) within the text to improve the AI's understanding of the conversation.
    \item \textbf{Dialogue Management}: Managing the flow of conversation to ensure coherent and contextually relevant interactions.
\end{itemize}

\section{Relevant Research Studies}

Several academic studies have explored AI and NLP in mental health, offering valuable insights:

\begin{itemize}
    \item \textbf{AI Chatbots for Anxiety and Depression}: Studies show that AI chatbots can help reduce symptoms of anxiety and depression by using cognitive-behavioral techniques, though effectiveness varies with user engagement and AI response quality.
    \item \textbf{Sentiment Analysis in Mental Health}: Research indicates that sentiment analysis can accurately assess emotional states in text data, aiding mental health monitoring. However, ensuring accuracy and context-awareness remains challenging.
    \item \textbf{User Acceptance of AI-Driven Tools}: Studies suggest that users are generally open to AI-driven mental health tools for initial support and self-help. However, data privacy, trust, and authenticity of AI responses are ongoing concerns.
    \item \textbf{Effectiveness of AI-Based Interventions}: Research has found varying levels of effectiveness for AI-based mental health interventions, often dependent on factors such as user engagement, the complexity of the mental health issue, and the quality of AI interactions.
\end{itemize}
% ------------------------------------------------------------------------------
