 ------------------------------------------------------------------------------
% Chapter 1
% ------------------------------------------------------------------------------
\chapter{Introduction} % enter the name of the chapter here
Mental health is a state of well-being that allows an individual to feel, think and act to their full potential. They are an integral part of health and well-being that underpin our individual and collective abilities to make decisions, establish relationships and shape the world in which we live. Mental health affects every aspect of our lives, including our physical health, our social relationships, our productivity at work or school, and our general sense of well-being. When our mental health is good, we are able to deal with daily challenges and feel happy and satisfied with life. However, mental health is an important issue that is often overlooked. In fact, according to the World Health Organization, about 1 in 4 people in the world will suffer from a mental disorder at some point in their lives. There are many reasons why people hesitate to seek help from psychiatrists. One reason is fear of stigma. There are still a lot of social judgments around mental health, and some people fear they will be seen as weak or unstable if they seek help. Another reason for hesitation is fear of cost. Mental health care can be expensive, especially in low- or middle-income countries. Finally, some people may find it difficult to talk about their feelings and experiences with a psychologist. They may feel ashamed or afraid of being judged. Therefore, we present the WANAS application, through which we provide an innovative solution to these challenges, where individuals can talk to artificial intelligence to obtain support and guidance at any time.
\section{ProjectIdea} % enter the name of the section here
The graduation project centers around the development of an application designed to function as a virtual therapist. The primary goal is to provide accessible and personalized mental health support in an increasingly fast-paced and interconnected world. The application utilizes artificial intelligence and machine learning to create an empathetic and intelligent virtual therapist capable of engaging with users in a meaningful way.

Key Features:

    Personalized Support: The virtual therapist is designed to offer tailored assistance, taking into account individual needs, preferences, and mental health concerns. Through advanced algorithms, the application adapts its responses to the user's specific emotional state and context.

    Confidential and Non-Judgmental Space: Recognizing the importance of privacy in mental health discussions, the application creates a secure and confidential environment. Users can freely express their thoughts and emotions without fear of judgment, fostering a sense of trust and openness.

    User-Friendly Interface: The project prioritizes a user-friendly interface to ensure that individuals, regardless of their technological proficiency, can easily navigate and interact with the virtual therapist. The goal is to make mental health support accessible to a broad audience.

    Integration of AI and Machine Learning: Leveraging the latest advancements in artificial intelligence and machine learning, the virtual therapist continuously learns and improves its responses over time. This dynamic adaptation ensures a more effective and personalized support system for users.

    Addressing Mental Health Stigma: The project aims to contribute to the destigmatization of mental health issues by providing a digital platform that encourages users to seek help in a discreet and comfortable manner.

Throughout the development process, ethical considerations are a key focus, ensuring user privacy, data security, and adherence to established mental health care guidelines. The project envisions making a meaningful impact on the mental well-being of individuals globally by offering a modern, technology-driven approach to mental health support.
 
\subsection{why?} %enter the name of the subsection here
The need for the virtual therapist application arises from several compelling reasons, addressing significant challenges in the current landscape of mental health care:

    Accessibility and Timeliness: Traditional mental health services often face challenges related to accessibility and timely support. The virtual therapist application fills this gap by providing immediate assistance, allowing users to access support whenever they need it, without the constraints of scheduling appointments or geographical limitations.

    Stigma Reduction: Mental health stigma remains a formidable barrier to seeking help. By offering a confidential and non-judgmental space, the application contributes to destigmatizing mental health issues. Users can engage with the virtual therapist discreetly, reducing the fear of social repercussions and promoting a more open dialogue about mental health.

    Global Reach: In a world where mental health issues are pervasive and diverse, a digital solution provides a scalable and globally accessible platform. The virtual therapist application has the potential to reach individuals in remote areas or regions with limited mental health resources, democratizing access to support services.

    Personalization and Adaptability: Every individual's mental health journey is unique, requiring personalized support. The application utilizes artificial intelligence and machine learning to adapt its responses to the specific needs and emotional states of users. This personalized approach enhances the effectiveness of the support provided.

    24/7 Support: Mental health concerns do not adhere to a fixed schedule. The virtual therapist application offers continuous support, 24 hours a day, 7 days a week. This ensures that users can seek assistance at any time, particularly during moments of crisis or when faced with urgent emotional challenges.

    Complementary Support to Traditional Services: The application is designed to complement traditional mental health services rather than replace them. It serves as an additional resource, offering immediate support and guidance, while also encouraging users to seek professional help when necessary.

    Technological Advancements: With the rapid advancement of technology, leveraging artificial intelligence and machine learning in mental health care represents a logical progression. The virtual therapist application harnesses these technologies to enhance the quality and accessibility of mental health support.

In summary, the virtual therapist application is a response to the evolving needs of individuals in a digital age, providing a convenient, stigma-free, and personalized solution to the challenges associated with mental health care. It aims to empower users to take control of their mental well-being by offering support that is accessible, adaptable, and considerate of the diverse nature of mental health experiences.




